
% Default to the notebook output style

    


% Inherit from the specified cell style.




    
\documentclass[12pt]{article}

    
    
    \usepackage[T1]{fontenc}
    % Nicer default font (+ math font) than Computer Modern for most use cases
    \usepackage{mathpazo}

    % Basic figure setup, for now with no caption control since it's done
    % automatically by Pandoc (which extracts ![](path) syntax from Markdown).
    \usepackage{graphicx}
    % We will generate all images so they have a width \maxwidth. This means
    % that they will get their normal width if they fit onto the page, but
    % are scaled down if they would overflow the margins.
    \makeatletter
    \def\maxwidth{\ifdim\Gin@nat@width>\linewidth\linewidth
    \else\Gin@nat@width\fi}
    \makeatother
    \let\Oldincludegraphics\includegraphics
    % Set max figure width to be 80% of text width, for now hardcoded.
    \renewcommand{\includegraphics}[1]{\Oldincludegraphics[width=.8\maxwidth]{#1}}
    % Ensure that by default, figures have no caption (until we provide a
    % proper Figure object with a Caption API and a way to capture that
    % in the conversion process - todo).
    \usepackage{caption}
    \DeclareCaptionLabelFormat{nolabel}{}
    \captionsetup{labelformat=nolabel}

    \usepackage{adjustbox} % Used to constrain images to a maximum size 
    \usepackage{xcolor} % Allow colors to be defined
    \usepackage{enumerate} % Needed for markdown enumerations to work
    \usepackage{geometry} % Used to adjust the document margins
    \usepackage{amsmath} % Equations
    \usepackage{amssymb} % Equations
    \usepackage{textcomp} % defines textquotesingle
    % Hack from http://tex.stackexchange.com/a/47451/13684:
    \AtBeginDocument{%
        \def\PYZsq{\textquotesingle}% Upright quotes in Pygmentized code
    }
    \usepackage{upquote} % Upright quotes for verbatim code
    \usepackage{eurosym} % defines \euro
    \usepackage[mathletters]{ucs} % Extended unicode (utf-8) support
    \usepackage[utf8x]{inputenc} % Allow utf-8 characters in the tex document
    \usepackage{fancyvrb} % verbatim replacement that allows latex
    \usepackage{grffile} % extends the file name processing of package graphics 
                         % to support a larger range 
    % The hyperref package gives us a pdf with properly built
    % internal navigation ('pdf bookmarks' for the table of contents,
    % internal cross-reference links, web links for URLs, etc.)
    \usepackage{hyperref}
    \usepackage{longtable} % longtable support required by pandoc >1.10
    \usepackage{booktabs}  % table support for pandoc > 1.12.2
    \usepackage[inline]{enumitem} % IRkernel/repr support (it uses the enumerate* environment)
    \usepackage[normalem]{ulem} % ulem is needed to support strikethroughs (\sout)
                                % normalem makes italics be italics, not underlines
    

    
    
    % Colors for the hyperref package
    \definecolor{urlcolor}{rgb}{0,.145,.698}
    \definecolor{linkcolor}{rgb}{.71,0.21,0.01}
    \definecolor{citecolor}{rgb}{.12,.54,.11}

    % ANSI colors
    \definecolor{ansi-black}{HTML}{3E424D}
    \definecolor{ansi-black-intense}{HTML}{282C36}
    \definecolor{ansi-red}{HTML}{E75C58}
    \definecolor{ansi-red-intense}{HTML}{B22B31}
    \definecolor{ansi-green}{HTML}{00A250}
    \definecolor{ansi-green-intense}{HTML}{007427}
    \definecolor{ansi-yellow}{HTML}{DDB62B}
    \definecolor{ansi-yellow-intense}{HTML}{B27D12}
    \definecolor{ansi-blue}{HTML}{208FFB}
    \definecolor{ansi-blue-intense}{HTML}{0065CA}
    \definecolor{ansi-magenta}{HTML}{D160C4}
    \definecolor{ansi-magenta-intense}{HTML}{A03196}
    \definecolor{ansi-cyan}{HTML}{60C6C8}
    \definecolor{ansi-cyan-intense}{HTML}{258F8F}
    \definecolor{ansi-white}{HTML}{C5C1B4}
    \definecolor{ansi-white-intense}{HTML}{A1A6B2}

    % commands and environments needed by pandoc snippets
    % extracted from the output of `pandoc -s`
    \providecommand{\tightlist}{%
      \setlength{\itemsep}{0pt}\setlength{\parskip}{0pt}}
    \DefineVerbatimEnvironment{Highlighting}{Verbatim}{commandchars=\\\{\}}
    % Add ',fontsize=\small' for more characters per line
    \newenvironment{Shaded}{}{}
    \newcommand{\KeywordTok}[1]{\textcolor[rgb]{0.00,0.44,0.13}{\textbf{{#1}}}}
    \newcommand{\DataTypeTok}[1]{\textcolor[rgb]{0.56,0.13,0.00}{{#1}}}
    \newcommand{\DecValTok}[1]{\textcolor[rgb]{0.25,0.63,0.44}{{#1}}}
    \newcommand{\BaseNTok}[1]{\textcolor[rgb]{0.25,0.63,0.44}{{#1}}}
    \newcommand{\FloatTok}[1]{\textcolor[rgb]{0.25,0.63,0.44}{{#1}}}
    \newcommand{\CharTok}[1]{\textcolor[rgb]{0.25,0.44,0.63}{{#1}}}
    \newcommand{\StringTok}[1]{\textcolor[rgb]{0.25,0.44,0.63}{{#1}}}
    \newcommand{\CommentTok}[1]{\textcolor[rgb]{0.38,0.63,0.69}{\textit{{#1}}}}
    \newcommand{\OtherTok}[1]{\textcolor[rgb]{0.00,0.44,0.13}{{#1}}}
    \newcommand{\AlertTok}[1]{\textcolor[rgb]{1.00,0.00,0.00}{\textbf{{#1}}}}
    \newcommand{\FunctionTok}[1]{\textcolor[rgb]{0.02,0.16,0.49}{{#1}}}
    \newcommand{\RegionMarkerTok}[1]{{#1}}
    \newcommand{\ErrorTok}[1]{\textcolor[rgb]{1.00,0.00,0.00}{\textbf{{#1}}}}
    \newcommand{\NormalTok}[1]{{#1}}
    
    % Additional commands for more recent versions of Pandoc
    \newcommand{\ConstantTok}[1]{\textcolor[rgb]{0.53,0.00,0.00}{{#1}}}
    \newcommand{\SpecialCharTok}[1]{\textcolor[rgb]{0.25,0.44,0.63}{{#1}}}
    \newcommand{\VerbatimStringTok}[1]{\textcolor[rgb]{0.25,0.44,0.63}{{#1}}}
    \newcommand{\SpecialStringTok}[1]{\textcolor[rgb]{0.73,0.40,0.53}{{#1}}}
    \newcommand{\ImportTok}[1]{{#1}}
    \newcommand{\DocumentationTok}[1]{\textcolor[rgb]{0.73,0.13,0.13}{\textit{{#1}}}}
    \newcommand{\AnnotationTok}[1]{\textcolor[rgb]{0.38,0.63,0.69}{\textbf{\textit{{#1}}}}}
    \newcommand{\CommentVarTok}[1]{\textcolor[rgb]{0.38,0.63,0.69}{\textbf{\textit{{#1}}}}}
    \newcommand{\VariableTok}[1]{\textcolor[rgb]{0.10,0.09,0.49}{{#1}}}
    \newcommand{\ControlFlowTok}[1]{\textcolor[rgb]{0.00,0.44,0.13}{\textbf{{#1}}}}
    \newcommand{\OperatorTok}[1]{\textcolor[rgb]{0.40,0.40,0.40}{{#1}}}
    \newcommand{\BuiltInTok}[1]{{#1}}
    \newcommand{\ExtensionTok}[1]{{#1}}
    \newcommand{\PreprocessorTok}[1]{\textcolor[rgb]{0.74,0.48,0.00}{{#1}}}
    \newcommand{\AttributeTok}[1]{\textcolor[rgb]{0.49,0.56,0.16}{{#1}}}
    \newcommand{\InformationTok}[1]{\textcolor[rgb]{0.38,0.63,0.69}{\textbf{\textit{{#1}}}}}
    \newcommand{\WarningTok}[1]{\textcolor[rgb]{0.38,0.63,0.69}{\textbf{\textit{{#1}}}}}
    
    
    % Define a nice break command that doesn't care if a line doesn't already
    % exist.
    \def\br{\hspace*{\fill} \\* }
    % Math Jax compatability definitions
    \def\gt{>}
    \def\lt{<}
    % Document parameters
    \title{P3 ICA Audio Signal Source Recovery}
    \author{William Koehrsen wjk68}
    \date{April 25, 2018}

    % Pygments definitions
    
\makeatletter
\def\PY@reset{\let\PY@it=\relax \let\PY@bf=\relax%
    \let\PY@ul=\relax \let\PY@tc=\relax%
    \let\PY@bc=\relax \let\PY@ff=\relax}
\def\PY@tok#1{\csname PY@tok@#1\endcsname}
\def\PY@toks#1+{\ifx\relax#1\empty\else%
    \PY@tok{#1}\expandafter\PY@toks\fi}
\def\PY@do#1{\PY@bc{\PY@tc{\PY@ul{%
    \PY@it{\PY@bf{\PY@ff{#1}}}}}}}
\def\PY#1#2{\PY@reset\PY@toks#1+\relax+\PY@do{#2}}

\expandafter\def\csname PY@tok@w\endcsname{\def\PY@tc##1{\textcolor[rgb]{0.73,0.73,0.73}{##1}}}
\expandafter\def\csname PY@tok@c\endcsname{\let\PY@it=\textit\def\PY@tc##1{\textcolor[rgb]{0.25,0.50,0.50}{##1}}}
\expandafter\def\csname PY@tok@cp\endcsname{\def\PY@tc##1{\textcolor[rgb]{0.74,0.48,0.00}{##1}}}
\expandafter\def\csname PY@tok@k\endcsname{\let\PY@bf=\textbf\def\PY@tc##1{\textcolor[rgb]{0.00,0.50,0.00}{##1}}}
\expandafter\def\csname PY@tok@kp\endcsname{\def\PY@tc##1{\textcolor[rgb]{0.00,0.50,0.00}{##1}}}
\expandafter\def\csname PY@tok@kt\endcsname{\def\PY@tc##1{\textcolor[rgb]{0.69,0.00,0.25}{##1}}}
\expandafter\def\csname PY@tok@o\endcsname{\def\PY@tc##1{\textcolor[rgb]{0.40,0.40,0.40}{##1}}}
\expandafter\def\csname PY@tok@ow\endcsname{\let\PY@bf=\textbf\def\PY@tc##1{\textcolor[rgb]{0.67,0.13,1.00}{##1}}}
\expandafter\def\csname PY@tok@nb\endcsname{\def\PY@tc##1{\textcolor[rgb]{0.00,0.50,0.00}{##1}}}
\expandafter\def\csname PY@tok@nf\endcsname{\def\PY@tc##1{\textcolor[rgb]{0.00,0.00,1.00}{##1}}}
\expandafter\def\csname PY@tok@nc\endcsname{\let\PY@bf=\textbf\def\PY@tc##1{\textcolor[rgb]{0.00,0.00,1.00}{##1}}}
\expandafter\def\csname PY@tok@nn\endcsname{\let\PY@bf=\textbf\def\PY@tc##1{\textcolor[rgb]{0.00,0.00,1.00}{##1}}}
\expandafter\def\csname PY@tok@ne\endcsname{\let\PY@bf=\textbf\def\PY@tc##1{\textcolor[rgb]{0.82,0.25,0.23}{##1}}}
\expandafter\def\csname PY@tok@nv\endcsname{\def\PY@tc##1{\textcolor[rgb]{0.10,0.09,0.49}{##1}}}
\expandafter\def\csname PY@tok@no\endcsname{\def\PY@tc##1{\textcolor[rgb]{0.53,0.00,0.00}{##1}}}
\expandafter\def\csname PY@tok@nl\endcsname{\def\PY@tc##1{\textcolor[rgb]{0.63,0.63,0.00}{##1}}}
\expandafter\def\csname PY@tok@ni\endcsname{\let\PY@bf=\textbf\def\PY@tc##1{\textcolor[rgb]{0.60,0.60,0.60}{##1}}}
\expandafter\def\csname PY@tok@na\endcsname{\def\PY@tc##1{\textcolor[rgb]{0.49,0.56,0.16}{##1}}}
\expandafter\def\csname PY@tok@nt\endcsname{\let\PY@bf=\textbf\def\PY@tc##1{\textcolor[rgb]{0.00,0.50,0.00}{##1}}}
\expandafter\def\csname PY@tok@nd\endcsname{\def\PY@tc##1{\textcolor[rgb]{0.67,0.13,1.00}{##1}}}
\expandafter\def\csname PY@tok@s\endcsname{\def\PY@tc##1{\textcolor[rgb]{0.73,0.13,0.13}{##1}}}
\expandafter\def\csname PY@tok@sd\endcsname{\let\PY@it=\textit\def\PY@tc##1{\textcolor[rgb]{0.73,0.13,0.13}{##1}}}
\expandafter\def\csname PY@tok@si\endcsname{\let\PY@bf=\textbf\def\PY@tc##1{\textcolor[rgb]{0.73,0.40,0.53}{##1}}}
\expandafter\def\csname PY@tok@se\endcsname{\let\PY@bf=\textbf\def\PY@tc##1{\textcolor[rgb]{0.73,0.40,0.13}{##1}}}
\expandafter\def\csname PY@tok@sr\endcsname{\def\PY@tc##1{\textcolor[rgb]{0.73,0.40,0.53}{##1}}}
\expandafter\def\csname PY@tok@ss\endcsname{\def\PY@tc##1{\textcolor[rgb]{0.10,0.09,0.49}{##1}}}
\expandafter\def\csname PY@tok@sx\endcsname{\def\PY@tc##1{\textcolor[rgb]{0.00,0.50,0.00}{##1}}}
\expandafter\def\csname PY@tok@m\endcsname{\def\PY@tc##1{\textcolor[rgb]{0.40,0.40,0.40}{##1}}}
\expandafter\def\csname PY@tok@gh\endcsname{\let\PY@bf=\textbf\def\PY@tc##1{\textcolor[rgb]{0.00,0.00,0.50}{##1}}}
\expandafter\def\csname PY@tok@gu\endcsname{\let\PY@bf=\textbf\def\PY@tc##1{\textcolor[rgb]{0.50,0.00,0.50}{##1}}}
\expandafter\def\csname PY@tok@gd\endcsname{\def\PY@tc##1{\textcolor[rgb]{0.63,0.00,0.00}{##1}}}
\expandafter\def\csname PY@tok@gi\endcsname{\def\PY@tc##1{\textcolor[rgb]{0.00,0.63,0.00}{##1}}}
\expandafter\def\csname PY@tok@gr\endcsname{\def\PY@tc##1{\textcolor[rgb]{1.00,0.00,0.00}{##1}}}
\expandafter\def\csname PY@tok@ge\endcsname{\let\PY@it=\textit}
\expandafter\def\csname PY@tok@gs\endcsname{\let\PY@bf=\textbf}
\expandafter\def\csname PY@tok@gp\endcsname{\let\PY@bf=\textbf\def\PY@tc##1{\textcolor[rgb]{0.00,0.00,0.50}{##1}}}
\expandafter\def\csname PY@tok@go\endcsname{\def\PY@tc##1{\textcolor[rgb]{0.53,0.53,0.53}{##1}}}
\expandafter\def\csname PY@tok@gt\endcsname{\def\PY@tc##1{\textcolor[rgb]{0.00,0.27,0.87}{##1}}}
\expandafter\def\csname PY@tok@err\endcsname{\def\PY@bc##1{\setlength{\fboxsep}{0pt}\fcolorbox[rgb]{1.00,0.00,0.00}{1,1,1}{\strut ##1}}}
\expandafter\def\csname PY@tok@kc\endcsname{\let\PY@bf=\textbf\def\PY@tc##1{\textcolor[rgb]{0.00,0.50,0.00}{##1}}}
\expandafter\def\csname PY@tok@kd\endcsname{\let\PY@bf=\textbf\def\PY@tc##1{\textcolor[rgb]{0.00,0.50,0.00}{##1}}}
\expandafter\def\csname PY@tok@kn\endcsname{\let\PY@bf=\textbf\def\PY@tc##1{\textcolor[rgb]{0.00,0.50,0.00}{##1}}}
\expandafter\def\csname PY@tok@kr\endcsname{\let\PY@bf=\textbf\def\PY@tc##1{\textcolor[rgb]{0.00,0.50,0.00}{##1}}}
\expandafter\def\csname PY@tok@bp\endcsname{\def\PY@tc##1{\textcolor[rgb]{0.00,0.50,0.00}{##1}}}
\expandafter\def\csname PY@tok@fm\endcsname{\def\PY@tc##1{\textcolor[rgb]{0.00,0.00,1.00}{##1}}}
\expandafter\def\csname PY@tok@vc\endcsname{\def\PY@tc##1{\textcolor[rgb]{0.10,0.09,0.49}{##1}}}
\expandafter\def\csname PY@tok@vg\endcsname{\def\PY@tc##1{\textcolor[rgb]{0.10,0.09,0.49}{##1}}}
\expandafter\def\csname PY@tok@vi\endcsname{\def\PY@tc##1{\textcolor[rgb]{0.10,0.09,0.49}{##1}}}
\expandafter\def\csname PY@tok@vm\endcsname{\def\PY@tc##1{\textcolor[rgb]{0.10,0.09,0.49}{##1}}}
\expandafter\def\csname PY@tok@sa\endcsname{\def\PY@tc##1{\textcolor[rgb]{0.73,0.13,0.13}{##1}}}
\expandafter\def\csname PY@tok@sb\endcsname{\def\PY@tc##1{\textcolor[rgb]{0.73,0.13,0.13}{##1}}}
\expandafter\def\csname PY@tok@sc\endcsname{\def\PY@tc##1{\textcolor[rgb]{0.73,0.13,0.13}{##1}}}
\expandafter\def\csname PY@tok@dl\endcsname{\def\PY@tc##1{\textcolor[rgb]{0.73,0.13,0.13}{##1}}}
\expandafter\def\csname PY@tok@s2\endcsname{\def\PY@tc##1{\textcolor[rgb]{0.73,0.13,0.13}{##1}}}
\expandafter\def\csname PY@tok@sh\endcsname{\def\PY@tc##1{\textcolor[rgb]{0.73,0.13,0.13}{##1}}}
\expandafter\def\csname PY@tok@s1\endcsname{\def\PY@tc##1{\textcolor[rgb]{0.73,0.13,0.13}{##1}}}
\expandafter\def\csname PY@tok@mb\endcsname{\def\PY@tc##1{\textcolor[rgb]{0.40,0.40,0.40}{##1}}}
\expandafter\def\csname PY@tok@mf\endcsname{\def\PY@tc##1{\textcolor[rgb]{0.40,0.40,0.40}{##1}}}
\expandafter\def\csname PY@tok@mh\endcsname{\def\PY@tc##1{\textcolor[rgb]{0.40,0.40,0.40}{##1}}}
\expandafter\def\csname PY@tok@mi\endcsname{\def\PY@tc##1{\textcolor[rgb]{0.40,0.40,0.40}{##1}}}
\expandafter\def\csname PY@tok@il\endcsname{\def\PY@tc##1{\textcolor[rgb]{0.40,0.40,0.40}{##1}}}
\expandafter\def\csname PY@tok@mo\endcsname{\def\PY@tc##1{\textcolor[rgb]{0.40,0.40,0.40}{##1}}}
\expandafter\def\csname PY@tok@ch\endcsname{\let\PY@it=\textit\def\PY@tc##1{\textcolor[rgb]{0.25,0.50,0.50}{##1}}}
\expandafter\def\csname PY@tok@cm\endcsname{\let\PY@it=\textit\def\PY@tc##1{\textcolor[rgb]{0.25,0.50,0.50}{##1}}}
\expandafter\def\csname PY@tok@cpf\endcsname{\let\PY@it=\textit\def\PY@tc##1{\textcolor[rgb]{0.25,0.50,0.50}{##1}}}
\expandafter\def\csname PY@tok@c1\endcsname{\let\PY@it=\textit\def\PY@tc##1{\textcolor[rgb]{0.25,0.50,0.50}{##1}}}
\expandafter\def\csname PY@tok@cs\endcsname{\let\PY@it=\textit\def\PY@tc##1{\textcolor[rgb]{0.25,0.50,0.50}{##1}}}

\def\PYZbs{\char`\\}
\def\PYZus{\char`\_}
\def\PYZob{\char`\{}
\def\PYZcb{\char`\}}
\def\PYZca{\char`\^}
\def\PYZam{\char`\&}
\def\PYZlt{\char`\<}
\def\PYZgt{\char`\>}
\def\PYZsh{\char`\#}
\def\PYZpc{\char`\%}
\def\PYZdl{\char`\$}
\def\PYZhy{\char`\-}
\def\PYZsq{\char`\'}
\def\PYZdq{\char`\"}
\def\PYZti{\char`\~}
% for compatibility with earlier versions
\def\PYZat{@}
\def\PYZlb{[}
\def\PYZrb{]}
\makeatother


    % Exact colors from NB
    \definecolor{incolor}{rgb}{0.0, 0.0, 0.5}
    \definecolor{outcolor}{rgb}{0.545, 0.0, 0.0}



    
    % Prevent overflowing lines due to hard-to-break entities
    \sloppy 
    % Setup hyperref package
    \hypersetup{
      breaklinks=true,  % so long urls are correctly broken across lines
      colorlinks=true,
      urlcolor=urlcolor,
      linkcolor=blue,
      citecolor=citecolor,
      }
    % Slightly bigger margins than the latex defaults
    
    \geometry{verbose,tmargin=1in,bmargin=1in,lmargin=1in,rmargin=1in}
    
    

    \begin{document}
    
    
    \maketitle
    \tableofcontents
    
    

    
    \hypertarget{introduction}{%
\section{Introduction: Audio Signal Source Recovery}\label{introduction}}

In this notebook we examine using the implementation of Independent
Component Analysis for separating mixed audio signals. Indepedent
component analysis (ICA) finds the sparse, independent components in a
signal under the assumption that the signal is a linear combination of
non-Gaussian components. ICA can be used for both blind source
separation, such as in this project, and dimensionality reduction. We
will verify the FastICA method developed in a previous notebook using
synthetic Laplacian data with a known mixing matrix. Then we will
attempt to separate out the sources and the mixing matrix for an audio
signal composed of two separate samples mixed with a random mixing
matrix.

The general equation governing ICA is that the signal is a linear
combination of samples:

\[X = AS\]

where X, the signal, is the product of the mixing matrix, A, and the
sample matrix, S. The objective is to separate out the indepedent
samples and the estimated mixing matrix. The specific implementation of
ICA used in this project is
\href{https://www.cs.helsinki.fi/u/ahyvarin/papers/TNN99new.pdf}{FastICA}
which has numerous advantages over gradient-based methods. The utility
function used in FastICA is
\href{https://www.cs.helsinki.fi/u/ahyvarin/papers/TR_A47_apprent.pdf}{Negentropy},
which is a measure of the non-Gaussianity of a distribution. The
objective is to maximize the negentropy of the independent components
given the data.

We will inspect the mixing matrix and indentified sources both visually
and quantitatively to determine if the implementation of ICA is working
correctly. We expect that for a signal composed of a linear combination
of non-Gaussian components, the ICA algorithm should be able to separate
out the sparse, independent components.

    \begin{Verbatim}[commandchars=\\\{\}]
{\color{incolor}In [{\color{incolor}1}]:} \PY{c+c1}{\PYZsh{} Pandas and numpy for data manipulation}
        \PY{k+kn}{import} \PY{n+nn}{pandas} \PY{k}{as} \PY{n+nn}{pd}
        \PY{k+kn}{import} \PY{n+nn}{numpy} \PY{k}{as} \PY{n+nn}{np}
\end{Verbatim}


    \begin{Verbatim}[commandchars=\\\{\}]
{\color{incolor}In [{\color{incolor}2}]:} \PY{c+c1}{\PYZsh{} Matploblit and seaborn for visualization}
        \PY{k+kn}{import} \PY{n+nn}{matplotlib}\PY{n+nn}{.}\PY{n+nn}{pyplot} \PY{k}{as} \PY{n+nn}{plt}
        \PY{o}{\PYZpc{}}\PY{k}{matplotlib} inline
        
        \PY{k+kn}{import} \PY{n+nn}{seaborn} \PY{k}{as} \PY{n+nn}{sns}
        
        \PY{c+c1}{\PYZsh{} Scipy io used for audio file management}
        \PY{k+kn}{from} \PY{n+nn}{scipy}\PY{n+nn}{.}\PY{n+nn}{io} \PY{k}{import} \PY{n}{wavfile}
        
        \PY{c+c1}{\PYZsh{} Scipy stats.mstats used for testing distributions}
        \PY{k+kn}{from} \PY{n+nn}{scipy}\PY{n+nn}{.}\PY{n+nn}{stats}\PY{n+nn}{.}\PY{n+nn}{mstats} \PY{k}{import} \PY{n}{normaltest}\PY{p}{,} \PY{n}{skewtest}\PY{p}{,} \PY{n}{kurtosistest}
\end{Verbatim}


    \hypertarget{import-fastica-algorithm}{%
\subsection{Import FastICA Algorithm}\label{import-fastica-algorithm}}

    \begin{Verbatim}[commandchars=\\\{\}]
{\color{incolor}In [{\color{incolor}3}]:} \PY{c+c1}{\PYZsh{} Import algorithm developed in previous notebook}
        \PY{k+kn}{from} \PY{n+nn}{parallel\PYZus{}ica} \PY{k}{import} \PY{n}{perform\PYZus{}fastica}
\end{Verbatim}


    \hypertarget{functions-from-demo}{%
\subsection{Functions from Demo}\label{functions-from-demo}}

    A number of functions were developed in the demo, primarily to help with
processing the audio data and drawing mixing matrices. We will use these
functions with a few minor modifications of the plotting format.

    \begin{Verbatim}[commandchars=\\\{\}]
{\color{incolor}In [{\color{incolor}4}]:} \PY{c+c1}{\PYZsh{} Normalize audio amplitude between \PYZhy{}0.5 and 0.5}
        \PY{k}{def} \PY{n+nf}{audionorm}\PY{p}{(}\PY{n}{data}\PY{p}{)}\PY{p}{:}
            \PY{c+c1}{\PYZsh{} ensure data is ndarray with float numbers}
            \PY{n}{data} \PY{o}{=} \PY{n}{np}\PY{o}{.}\PY{n}{asarray}\PY{p}{(}\PY{n}{data}\PY{p}{)}\PY{o}{.}\PY{n}{astype}\PY{p}{(}\PY{l+s+s1}{\PYZsq{}}\PY{l+s+s1}{float}\PY{l+s+s1}{\PYZsq{}}\PY{p}{)}
            \PY{c+c1}{\PYZsh{} calculate lower and upper bound}
            \PY{n}{lbound}\PY{p}{,} \PY{n}{ubound} \PY{o}{=} \PY{n}{np}\PY{o}{.}\PY{n}{min}\PY{p}{(}\PY{n}{data}\PY{p}{)}\PY{p}{,} \PY{n}{np}\PY{o}{.}\PY{n}{max}\PY{p}{(}\PY{n}{data}\PY{p}{)}
            \PY{k}{if} \PY{n}{lbound} \PY{o}{==} \PY{n}{ubound}\PY{p}{:}
                \PY{n}{offset} \PY{o}{=} \PY{n}{lbound}
                \PY{n}{scalar} \PY{o}{=} \PY{l+m+mi}{1}
                \PY{n}{data} \PY{o}{=} \PY{n}{np}\PY{o}{.}\PY{n}{zeros}\PY{p}{(}\PY{n}{size}\PY{o}{=}\PY{n}{data}\PY{o}{.}\PY{n}{shape}\PY{p}{)}
            \PY{k}{else}\PY{p}{:}
                \PY{n}{offset} \PY{o}{=} \PY{p}{(}\PY{n}{lbound} \PY{o}{+} \PY{n}{ubound}\PY{p}{)} \PY{o}{/} \PY{l+m+mi}{2}
                \PY{n}{scalar} \PY{o}{=} \PY{l+m+mi}{1} \PY{o}{/} \PY{p}{(}\PY{n}{ubound} \PY{o}{\PYZhy{}} \PY{n}{lbound}\PY{p}{)}
                \PY{n}{data} \PY{o}{=} \PY{p}{(}\PY{n}{data} \PY{o}{\PYZhy{}} \PY{n}{offset}\PY{p}{)} \PY{o}{*} \PY{n}{scalar}
            \PY{c+c1}{\PYZsh{} return normalized data}
            \PY{k}{return} \PY{n}{data}
        
        \PY{c+c1}{\PYZsh{} Mix a matrix of samples using a random mixing matrix}
        \PY{k}{def} \PY{n+nf}{simpleMixer}\PY{p}{(}\PY{n}{S}\PY{p}{)}\PY{p}{:}
            \PY{c+c1}{\PYZsh{} Set seed for consistent runs}
            \PY{n}{np}\PY{o}{.}\PY{n}{random}\PY{o}{.}\PY{n}{seed}\PY{p}{(}\PY{l+m+mi}{42}\PY{p}{)}
            \PY{n}{nchannel} \PY{o}{=} \PY{n}{S}\PY{o}{.}\PY{n}{shape}\PY{p}{[}\PY{l+m+mi}{0}\PY{p}{]}
            \PY{c+c1}{\PYZsh{} generate a random matrix}
            \PY{n}{A} \PY{o}{=} \PY{n}{np}\PY{o}{.}\PY{n}{random}\PY{o}{.}\PY{n}{uniform}\PY{p}{(}\PY{n}{size} \PY{o}{=} \PY{p}{(}\PY{n}{nchannel}\PY{p}{,}\PY{n}{nchannel}\PY{p}{)}\PY{p}{)}
            \PY{c+c1}{\PYZsh{} generate mixed audio data}
            \PY{n}{X} \PY{o}{=} \PY{n}{A}\PY{o}{.}\PY{n}{dot}\PY{p}{(}\PY{n}{S}\PY{p}{)}
            
            \PY{k}{return} \PY{n}{X}\PY{p}{,} \PY{n}{A}
        
        \PY{c+c1}{\PYZsh{} Plot the mixed signal with the mixing matrix}
        \PY{k}{def} \PY{n+nf}{drawDataWithMixingMatrix}\PY{p}{(}\PY{n}{data}\PY{p}{,} \PY{n}{mat}\PY{p}{,} \PY{n}{ground\PYZus{}truth}\PY{o}{=}\PY{k+kc}{True}\PY{p}{)}\PY{p}{:}
            \PY{n}{plt}\PY{o}{.}\PY{n}{figure}\PY{p}{(}\PY{n}{figsize}\PY{o}{=}\PY{p}{(}\PY{l+m+mi}{8}\PY{p}{,} \PY{l+m+mi}{6}\PY{p}{)}\PY{p}{)}
            \PY{c+c1}{\PYZsh{} plot data points}
            \PY{n}{plt}\PY{o}{.}\PY{n}{scatter}\PY{p}{(}\PY{n}{data}\PY{p}{[}\PY{l+m+mi}{0}\PY{p}{]}\PY{p}{,} \PY{n}{data}\PY{p}{[}\PY{l+m+mi}{1}\PY{p}{]}\PY{p}{,} \PY{n}{s}\PY{o}{=}\PY{l+m+mi}{2}\PY{p}{,} \PY{n}{alpha} \PY{o}{=} \PY{l+m+mf}{0.6}\PY{p}{)}
            \PY{c+c1}{\PYZsh{} calculate axis length}
            \PY{n}{lenAxis} \PY{o}{=} \PY{n}{np}\PY{o}{.}\PY{n}{sqrt}\PY{p}{(}\PY{n}{np}\PY{o}{.}\PY{n}{sum}\PY{p}{(}\PY{n}{np}\PY{o}{.}\PY{n}{square}\PY{p}{(}\PY{n}{mat}\PY{p}{)}\PY{p}{,} \PY{n}{axis}\PY{o}{=}\PY{l+m+mi}{0}\PY{p}{)}\PY{p}{)}
            \PY{c+c1}{\PYZsh{} calculate scale for illustration}
            \PY{n}{scale} \PY{o}{=} \PY{n}{np}\PY{o}{.}\PY{n}{min}\PY{p}{(}\PY{n}{np}\PY{o}{.}\PY{n}{max}\PY{p}{(}\PY{n}{np}\PY{o}{.}\PY{n}{abs}\PY{p}{(}\PY{n}{data}\PY{p}{)}\PY{p}{,} \PY{n}{axis}\PY{o}{=}\PY{l+m+mi}{1}\PY{p}{)} \PY{o}{/} \PY{n}{lenAxis}\PY{o}{.}\PY{n}{T}\PY{p}{)}
            \PY{c+c1}{\PYZsh{} draw axis as arrow}
            \PY{n}{plt}\PY{o}{.}\PY{n}{arrow}\PY{p}{(}\PY{l+m+mi}{0}\PY{p}{,} \PY{l+m+mi}{0}\PY{p}{,} \PY{n}{scale} \PY{o}{*} \PY{n}{mat}\PY{p}{[}\PY{l+m+mi}{0}\PY{p}{,}\PY{l+m+mi}{0}\PY{p}{]}\PY{p}{,} \PY{n}{scale} \PY{o}{*} \PY{n}{mat}\PY{p}{[}\PY{l+m+mi}{1}\PY{p}{,}\PY{l+m+mi}{0}\PY{p}{]}\PY{p}{,} \PY{n}{shape}\PY{o}{=}\PY{l+s+s1}{\PYZsq{}}\PY{l+s+s1}{full}\PY{l+s+s1}{\PYZsq{}}\PY{p}{,} \PY{n}{color}\PY{o}{=}\PY{l+s+s1}{\PYZsq{}}\PY{l+s+s1}{r}\PY{l+s+s1}{\PYZsq{}}\PY{p}{)}
            \PY{n}{plt}\PY{o}{.}\PY{n}{arrow}\PY{p}{(}\PY{l+m+mi}{0}\PY{p}{,} \PY{l+m+mi}{0}\PY{p}{,} \PY{n}{scale} \PY{o}{*} \PY{n}{mat}\PY{p}{[}\PY{l+m+mi}{0}\PY{p}{,}\PY{l+m+mi}{1}\PY{p}{]}\PY{p}{,} \PY{n}{scale} \PY{o}{*} \PY{n}{mat}\PY{p}{[}\PY{l+m+mi}{1}\PY{p}{,}\PY{l+m+mi}{1}\PY{p}{]}\PY{p}{,} \PY{n}{shape}\PY{o}{=}\PY{l+s+s1}{\PYZsq{}}\PY{l+s+s1}{full}\PY{l+s+s1}{\PYZsq{}}\PY{p}{,} \PY{n}{color}\PY{o}{=}\PY{l+s+s1}{\PYZsq{}}\PY{l+s+s1}{r}\PY{l+s+s1}{\PYZsq{}}\PY{p}{)}
            \PY{k}{if} \PY{n}{ground\PYZus{}truth}\PY{p}{:}
                \PY{n}{plt}\PY{o}{.}\PY{n}{title}\PY{p}{(}\PY{l+s+s1}{\PYZsq{}}\PY{l+s+s1}{Data with True Mixing Matrix}\PY{l+s+s1}{\PYZsq{}}\PY{p}{,} \PY{n}{size} \PY{o}{=} \PY{l+m+mi}{18}\PY{p}{)}
            \PY{k}{else}\PY{p}{:}
                \PY{n}{plt}\PY{o}{.}\PY{n}{title}\PY{p}{(}\PY{l+s+s1}{\PYZsq{}}\PY{l+s+s1}{Data with Estimated Mixing Matrix}\PY{l+s+s1}{\PYZsq{}}\PY{p}{,} \PY{n}{size} \PY{o}{=} \PY{l+m+mi}{18}\PY{p}{)}
            
        \PY{c+c1}{\PYZsh{} Show mixed signal with true mixing matrix and estimated mixing matrix}
        \PY{k}{def} \PY{n+nf}{compareMixingMatrix}\PY{p}{(}\PY{n}{data}\PY{p}{,} \PY{n}{matA}\PY{p}{,} \PY{n}{matB}\PY{p}{)}\PY{p}{:}
            \PY{c+c1}{\PYZsh{} plot first mixing matrix}
            \PY{n}{drawDataWithMixingMatrix}\PY{p}{(}\PY{n}{data}\PY{p}{,} \PY{n}{matA}\PY{p}{)}
            \PY{c+c1}{\PYZsh{} plot first mixing matrix}
            \PY{n}{drawDataWithMixingMatrix}\PY{p}{(}\PY{n}{data}\PY{p}{,} \PY{n}{matB}\PY{p}{,} \PY{n}{ground\PYZus{}truth}\PY{o}{=}\PY{k+kc}{False}\PY{p}{)}
\end{Verbatim}


    \hypertarget{read-in-audio-data-and-visualize}{%
\subsubsection{Read in Audio Data and
Visualize}\label{read-in-audio-data-and-visualize}}

    The first audio sample is a piece of Bach music while the second is a
snippet of speech. The audio files are normed between {[}-0.5, 0.5{]}
and plotted below. We can see that each sample is slightly more than
120,000 separate points, with each point representing an amplitude.

    \begin{Verbatim}[commandchars=\\\{\}]
{\color{incolor}In [{\color{incolor}5}]:} \PY{c+c1}{\PYZsh{} Bach Audio}
        \PY{n}{srate}\PY{p}{,} \PY{n}{bach\PYZus{}sample} \PY{o}{=} \PY{n}{wavfile}\PY{o}{.}\PY{n}{read}\PY{p}{(}\PY{l+s+s1}{\PYZsq{}}\PY{l+s+s1}{../data/bach.wav}\PY{l+s+s1}{\PYZsq{}}\PY{p}{)}
        \PY{n}{bach\PYZus{}sample} \PY{o}{=} \PY{n}{audionorm}\PY{p}{(}\PY{n}{bach\PYZus{}sample}\PY{p}{)}
        \PY{n}{plt}\PY{o}{.}\PY{n}{figure}\PY{p}{(}\PY{n}{figsize}\PY{o}{=}\PY{p}{(}\PY{l+m+mi}{7}\PY{p}{,} \PY{l+m+mi}{5}\PY{p}{)}\PY{p}{)}
        \PY{n}{plt}\PY{o}{.}\PY{n}{plot}\PY{p}{(}\PY{n}{bach\PYZus{}sample}\PY{p}{,} \PY{n}{color} \PY{o}{=} \PY{l+s+s1}{\PYZsq{}}\PY{l+s+s1}{orange}\PY{l+s+s1}{\PYZsq{}}\PY{p}{)}
        \PY{n}{plt}\PY{o}{.}\PY{n}{title}\PY{p}{(}\PY{l+s+s1}{\PYZsq{}}\PY{l+s+s1}{Normed Bach Sample}\PY{l+s+s1}{\PYZsq{}}\PY{p}{)}\PY{p}{;}
        \PY{n}{plt}\PY{o}{.}\PY{n}{show}\PY{p}{(}\PY{p}{)}\PY{p}{;}
        
        
        \PY{c+c1}{\PYZsh{} Speech audio}
        \PY{n}{\PYZus{}}\PY{p}{,} \PY{n}{speech\PYZus{}sample} \PY{o}{=} \PY{n}{wavfile}\PY{o}{.}\PY{n}{read}\PY{p}{(}\PY{l+s+s1}{\PYZsq{}}\PY{l+s+s1}{../data/speech.wav}\PY{l+s+s1}{\PYZsq{}}\PY{p}{)}
        \PY{n}{speech\PYZus{}sample} \PY{o}{=} \PY{n}{audionorm}\PY{p}{(}\PY{n}{speech\PYZus{}sample}\PY{p}{)}
        \PY{n}{plt}\PY{o}{.}\PY{n}{figure}\PY{p}{(}\PY{n}{figsize}\PY{o}{=}\PY{p}{(}\PY{l+m+mi}{7}\PY{p}{,} \PY{l+m+mi}{5}\PY{p}{)}\PY{p}{)}
        \PY{n}{plt}\PY{o}{.}\PY{n}{plot}\PY{p}{(}\PY{n}{speech\PYZus{}sample}\PY{p}{,} \PY{n}{color} \PY{o}{=} \PY{l+s+s1}{\PYZsq{}}\PY{l+s+s1}{green}\PY{l+s+s1}{\PYZsq{}}\PY{p}{)}
        \PY{n}{plt}\PY{o}{.}\PY{n}{title}\PY{p}{(}\PY{l+s+s1}{\PYZsq{}}\PY{l+s+s1}{Normed Speech Sample}\PY{l+s+s1}{\PYZsq{}}\PY{p}{)}\PY{p}{;}
        \PY{n}{plt}\PY{o}{.}\PY{n}{show}\PY{p}{(}\PY{p}{)}\PY{p}{;}
\end{Verbatim}


    \begin{center}
    \adjustimage{max size={0.9\linewidth}{0.9\paperheight}}{output_10_0.png}
    \end{center}
    { \hspace*{\fill} \\}
    
    \begin{center}
    \adjustimage{max size={0.9\linewidth}{0.9\paperheight}}{output_10_1.png}
    \end{center}
    { \hspace*{\fill} \\}
    
    \hypertarget{check-gaussianity-of-audio-data}{%
\section{Check Gaussianity of Audio
Data}\label{check-gaussianity-of-audio-data}}

Independent component analysis works on the assumption that the
components of a signal are non-Gaussian. To test this, we can calculate
the skew and kurtosis as well as the D'Agostino and Pearson's Test that
combines the two measures to arrive at a single omnibus estimate for
Gaussianity of the data. We will use
\href{https://docs.scipy.org/doc/scipy-0.14.0/reference/generated/scipy.stats.mstats.normaltest.html}{\texttt{scipy.stats.mstats.normaltest}}
which tests the Null Hypothesis that the sample is from a Gaussian
distribution. If the p-value is less than 0.05, we can reject the null
hypothesis, and conclude there is evidence to prove that the data is not
Gaussian. We can also plot the histogram of the data along with a
Gaussian and Laplacian distribution of the same size for a visualize
comparison.

The normal test calculates the \(k2\) measures defined as
\(k2 = s^2 + k^2\) where s is the skew of the distribution and k is the
kurtosis. The test for normality was developed by D'Agostino and Pearson
``D'Agostino, R. and Pearson, E. S. (1973), ``Testing for departures
from normality,'' Biometrika, 60, 613-622''. To get a sense of the skew
and kurtosis measures, we can calculate them for the signal, a Gaussian
distribution, and a Laplacian distribution. Combining the visual with
the quantitative stats is a helpful way to think through concepts.

    \begin{Verbatim}[commandchars=\\\{\}]
{\color{incolor}In [{\color{incolor}6}]:} \PY{k}{def} \PY{n+nf}{check\PYZus{}gaussianity}\PY{p}{(}\PY{n}{X}\PY{p}{)}\PY{p}{:}
            \PY{c+c1}{\PYZsh{} Perform D\PYZsq{}Agostino and Pearson\PYZsq{}s test to determine normality}
            \PY{c+c1}{\PYZsh{} Combines skew and kurtosis for combined test of normality}
            \PY{c+c1}{\PYZsh{} k2 = s\PYZca{}2 + k\PYZca{}2 where s is skew and k is kurtosis}
            \PY{n}{test\PYZus{}result} \PY{o}{=} \PY{n}{normaltest}\PY{p}{(}\PY{n}{X}\PY{p}{)}
            \PY{n+nb}{print}\PY{p}{(}\PY{l+s+s1}{\PYZsq{}}\PY{l+s+s1}{Normal Test p\PYZhy{}value = }\PY{l+s+si}{\PYZob{}:0.4f\PYZcb{}}\PY{l+s+s1}{\PYZsq{}}\PY{o}{.}\PY{n}{format}\PY{p}{(}\PY{n}{test\PYZus{}result}\PY{o}{.}\PY{n}{pvalue}\PY{p}{)}\PY{p}{)}
            
            \PY{c+c1}{\PYZsh{} Calculate actual skew and kurtosis}
            \PY{n}{skew} \PY{o}{=} \PY{n}{skewtest}\PY{p}{(}\PY{n}{X}\PY{p}{)}\PY{o}{.}\PY{n}{statistic}
            \PY{n}{kurtosis} \PY{o}{=} \PY{n}{kurtosistest}\PY{p}{(}\PY{n}{X}\PY{p}{)}\PY{o}{.}\PY{n}{statistic}
            
            \PY{c+c1}{\PYZsh{} Find Gaussian and Laplacian Distributions of same length}
            \PY{n}{gaussian} \PY{o}{=} \PY{n}{np}\PY{o}{.}\PY{n}{random}\PY{o}{.}\PY{n}{randn}\PY{p}{(}\PY{n+nb}{len}\PY{p}{(}\PY{n}{X}\PY{p}{)}\PY{p}{)}
            \PY{n}{laplacian} \PY{o}{=} \PY{n}{np}\PY{o}{.}\PY{n}{random}\PY{o}{.}\PY{n}{laplace}\PY{p}{(}\PY{n}{size} \PY{o}{=} \PY{n+nb}{len}\PY{p}{(}\PY{n}{X}\PY{p}{)}\PY{p}{)}
            
            \PY{c+c1}{\PYZsh{} Interpret p\PYZhy{}value}
            \PY{k}{if} \PY{n}{test\PYZus{}result}\PY{o}{.}\PY{n}{pvalue} \PY{o}{\PYZlt{}} \PY{l+m+mf}{0.05}\PY{p}{:}
                \PY{n+nb}{print}\PY{p}{(}\PY{l+s+s1}{\PYZsq{}}\PY{l+s+s1}{Reject Null Hypothesis that sample comes}\PY{l+s+s1}{\PYZsq{}}
                      \PY{l+s+s1}{\PYZsq{}}\PY{l+s+s1}{ from a normal distribution at alpha = 0.05}\PY{l+s+s1}{\PYZsq{}}\PY{p}{)}
            \PY{k}{else}\PY{p}{:}
                \PY{n+nb}{print}\PY{p}{(}\PY{l+s+s1}{\PYZsq{}}\PY{l+s+s1}{Fail to reject Null Hypothesis that sample}\PY{l+s+s1}{\PYZsq{}}
                      \PY{l+s+s1}{\PYZsq{}}\PY{l+s+s1}{ comes from a normal distribution at alpha= 0.05}\PY{l+s+s1}{\PYZsq{}}\PY{p}{)}
                
                
            \PY{n+nb}{print}\PY{p}{(}\PY{l+s+s1}{\PYZsq{}}\PY{l+s+se}{\PYZbs{}n}\PY{l+s+s1}{Sample Distribution Skew = }\PY{l+s+si}{\PYZob{}:0.4f\PYZcb{}}\PY{l+s+s1}{\PYZsq{}}
                      \PY{l+s+s1}{\PYZsq{}}\PY{l+s+s1}{  Kurtosis = }\PY{l+s+si}{\PYZob{}:0.4f\PYZcb{}}\PY{l+s+s1}{\PYZsq{}}\PY{o}{.}\PY{n}{format}\PY{p}{(}\PY{n}{skew}\PY{p}{,} \PY{n}{kurtosis}\PY{p}{)}\PY{p}{)}
            
            \PY{n+nb}{print}\PY{p}{(}\PY{l+s+s1}{\PYZsq{}}\PY{l+s+s1}{Gaussian Distribution Skew = }\PY{l+s+si}{\PYZob{}:0.4f\PYZcb{}}\PY{l+s+s1}{\PYZsq{}}
                      \PY{l+s+s1}{\PYZsq{}}\PY{l+s+s1}{  Kurtosis = }\PY{l+s+si}{\PYZob{}:0.4f\PYZcb{}}\PY{l+s+s1}{\PYZsq{}}\PY{o}{.}\PY{n}{format}\PY{p}{(}\PY{n}{skewtest}\PY{p}{(}\PY{n}{gaussian}\PY{p}{)}\PY{o}{.}\PY{n}{statistic}\PY{p}{,} 
                                                    \PY{n}{kurtosistest}\PY{p}{(}\PY{n}{gaussian}\PY{p}{)}\PY{o}{.}\PY{n}{statistic}\PY{p}{)}\PY{p}{)}
            \PY{n+nb}{print}\PY{p}{(}\PY{l+s+s1}{\PYZsq{}}\PY{l+s+s1}{Laplacian Distribution Skew = }\PY{l+s+si}{\PYZob{}:0.4f\PYZcb{}}\PY{l+s+s1}{\PYZsq{}}
                      \PY{l+s+s1}{\PYZsq{}}\PY{l+s+s1}{  Kurtosis = }\PY{l+s+si}{\PYZob{}:0.4f\PYZcb{}}\PY{l+s+se}{\PYZbs{}n}\PY{l+s+s1}{\PYZsq{}}\PY{o}{.}\PY{n}{format}\PY{p}{(}\PY{n}{skewtest}\PY{p}{(}\PY{n}{laplacian}\PY{p}{)}\PY{o}{.}\PY{n}{statistic}\PY{p}{,} 
                                                    \PY{n}{kurtosistest}\PY{p}{(}\PY{n}{laplacian}\PY{p}{)}\PY{o}{.}\PY{n}{statistic}\PY{p}{)}\PY{p}{)}
            
            
            \PY{n}{plt}\PY{o}{.}\PY{n}{figure}\PY{p}{(}\PY{n}{figsize}\PY{o}{=}\PY{p}{(}\PY{l+m+mi}{16}\PY{p}{,} \PY{l+m+mi}{8}\PY{p}{)}\PY{p}{)}
            
            \PY{c+c1}{\PYZsh{} Plot sample distribution}
            \PY{n}{plt}\PY{o}{.}\PY{n}{subplot}\PY{p}{(}\PY{l+m+mi}{131}\PY{p}{)}
            \PY{n}{sns}\PY{o}{.}\PY{n}{distplot}\PY{p}{(}\PY{n}{X}\PY{p}{,} \PY{n}{kde}\PY{o}{=}\PY{k+kc}{True}\PY{p}{,} \PY{n}{hist}\PY{o}{=}\PY{k+kc}{True}\PY{p}{,} \PY{n}{bins}\PY{o}{=}\PY{l+m+mi}{30}\PY{p}{,} \PY{n}{color} \PY{o}{=} \PY{l+s+s1}{\PYZsq{}}\PY{l+s+s1}{blue}\PY{l+s+s1}{\PYZsq{}}\PY{p}{)}
            \PY{n}{plt}\PY{o}{.}\PY{n}{xlabel}\PY{p}{(}\PY{l+s+s1}{\PYZsq{}}\PY{l+s+s1}{Amplitude}\PY{l+s+s1}{\PYZsq{}}\PY{p}{,} \PY{n}{size} \PY{o}{=} \PY{l+m+mi}{14}\PY{p}{)}\PY{p}{;} \PY{n}{plt}\PY{o}{.}\PY{n}{ylabel}\PY{p}{(}\PY{l+s+s1}{\PYZsq{}}\PY{l+s+s1}{Density}\PY{l+s+s1}{\PYZsq{}}\PY{p}{,} \PY{n}{size} \PY{o}{=} \PY{l+m+mi}{14}\PY{p}{)}\PY{p}{;}
            \PY{n}{plt}\PY{o}{.}\PY{n}{title}\PY{p}{(}\PY{l+s+s1}{\PYZsq{}}\PY{l+s+s1}{Sample Distribution}\PY{l+s+s1}{\PYZsq{}}\PY{p}{,} \PY{n}{size} \PY{o}{=} \PY{l+m+mi}{18}\PY{p}{)}\PY{p}{;}
            
            \PY{c+c1}{\PYZsh{} Plot Gaussian Distribution of same length}
            \PY{n}{plt}\PY{o}{.}\PY{n}{subplot}\PY{p}{(}\PY{l+m+mi}{132}\PY{p}{)}
            \PY{n}{sns}\PY{o}{.}\PY{n}{distplot}\PY{p}{(}\PY{n}{gaussian}\PY{p}{,} \PY{n}{kde}\PY{o}{=}\PY{k+kc}{True}\PY{p}{,} \PY{n}{hist}\PY{o}{=}\PY{k+kc}{True}\PY{p}{,} \PY{n}{bins}\PY{o}{=}\PY{l+m+mi}{30}\PY{p}{,} \PY{n}{color} \PY{o}{=} \PY{l+s+s1}{\PYZsq{}}\PY{l+s+s1}{blue}\PY{l+s+s1}{\PYZsq{}}\PY{p}{)}
            \PY{n}{plt}\PY{o}{.}\PY{n}{xlabel}\PY{p}{(}\PY{l+s+s1}{\PYZsq{}}\PY{l+s+s1}{Amplitude}\PY{l+s+s1}{\PYZsq{}}\PY{p}{,} \PY{n}{size} \PY{o}{=} \PY{l+m+mi}{14}\PY{p}{)}\PY{p}{;} \PY{n}{plt}\PY{o}{.}\PY{n}{ylabel}\PY{p}{(}\PY{l+s+s1}{\PYZsq{}}\PY{l+s+s1}{Density}\PY{l+s+s1}{\PYZsq{}}\PY{p}{,} \PY{n}{size} \PY{o}{=} \PY{l+m+mi}{14}\PY{p}{)}\PY{p}{;}
            \PY{n}{plt}\PY{o}{.}\PY{n}{title}\PY{p}{(}\PY{l+s+s1}{\PYZsq{}}\PY{l+s+s1}{Gaussian Distribution}\PY{l+s+s1}{\PYZsq{}}\PY{p}{,} \PY{n}{size} \PY{o}{=} \PY{l+m+mi}{18}\PY{p}{)}\PY{p}{;}
            
            \PY{c+c1}{\PYZsh{} Plot laplace}
            \PY{n}{plt}\PY{o}{.}\PY{n}{subplot}\PY{p}{(}\PY{l+m+mi}{133}\PY{p}{)}
            \PY{n}{sns}\PY{o}{.}\PY{n}{distplot}\PY{p}{(}\PY{n}{laplacian}\PY{p}{,} \PY{n}{kde}\PY{o}{=}\PY{k+kc}{True}\PY{p}{,} \PY{n}{hist}\PY{o}{=}\PY{k+kc}{True}\PY{p}{,} \PY{n}{bins}\PY{o}{=}\PY{l+m+mi}{30}\PY{p}{,} \PY{n}{color} \PY{o}{=} \PY{l+s+s1}{\PYZsq{}}\PY{l+s+s1}{blue}\PY{l+s+s1}{\PYZsq{}}\PY{p}{)}
            \PY{n}{plt}\PY{o}{.}\PY{n}{xlabel}\PY{p}{(}\PY{l+s+s1}{\PYZsq{}}\PY{l+s+s1}{Amplitude}\PY{l+s+s1}{\PYZsq{}}\PY{p}{,} \PY{n}{size} \PY{o}{=} \PY{l+m+mi}{18}\PY{p}{)}\PY{p}{;} \PY{n}{plt}\PY{o}{.}\PY{n}{ylabel}\PY{p}{(}\PY{l+s+s1}{\PYZsq{}}\PY{l+s+s1}{Density}\PY{l+s+s1}{\PYZsq{}}\PY{p}{,} \PY{n}{size} \PY{o}{=} \PY{l+m+mi}{18}\PY{p}{)}\PY{p}{;}
            \PY{n}{plt}\PY{o}{.}\PY{n}{title}\PY{p}{(}\PY{l+s+s1}{\PYZsq{}}\PY{l+s+s1}{Laplacian Distribution}\PY{l+s+s1}{\PYZsq{}}\PY{p}{,} \PY{n}{size} \PY{o}{=} \PY{l+m+mi}{18}\PY{p}{)}\PY{p}{;}
            
            \PY{n}{plt}\PY{o}{.}\PY{n}{show}\PY{p}{(}\PY{p}{)}\PY{p}{;}
            
\end{Verbatim}


    \hypertarget{test-check-gaussianity-function}{%
\subsection{Test Check Gaussianity
Function}\label{test-check-gaussianity-function}}

First, we will just check the Gaussianity testing function on a dataset
that we know is Gaussian, one generated by \texttt{np.random.randn}. The
p-value should be much greater than 0.05, indicating that we should fail
to reject the null hypothesis that the data is from a Gaussian
distribution.

    \begin{Verbatim}[commandchars=\\\{\}]
{\color{incolor}In [{\color{incolor}7}]:} \PY{n}{check\PYZus{}gaussianity}\PY{p}{(}\PY{n}{np}\PY{o}{.}\PY{n}{random}\PY{o}{.}\PY{n}{randn}\PY{p}{(}\PY{n+nb}{len}\PY{p}{(}\PY{n}{bach\PYZus{}sample}\PY{p}{)}\PY{p}{)}\PY{p}{)}
\end{Verbatim}


    \begin{Verbatim}[commandchars=\\\{\}]
Normal Test p-value = 0.3822
Fail to reject Null Hypothesis that sample comes from a normal distribution at alpha= 0.05

Sample Distribution Skew = 0.6815  Kurtosis = 1.2080
Gaussian Distribution Skew = 2.0214  Kurtosis = -0.9293
Laplacian Distribution Skew = 5.0398  Kurtosis = 95.5442


    \end{Verbatim}

    \begin{center}
    \adjustimage{max size={0.9\linewidth}{0.9\paperheight}}{output_14_1.png}
    \end{center}
    { \hspace*{\fill} \\}
    
    The check Gaussianity function clearly identifies that this is a
Gaussian Distribution.

    \hypertarget{test-bach-audio-sample}{%
\subsection{Test Bach Audio Sample}\label{test-bach-audio-sample}}

    \begin{Verbatim}[commandchars=\\\{\}]
{\color{incolor}In [{\color{incolor}8}]:} \PY{n}{check\PYZus{}gaussianity}\PY{p}{(}\PY{n}{bach\PYZus{}sample}\PY{p}{)}
\end{Verbatim}


    \begin{Verbatim}[commandchars=\\\{\}]
Normal Test p-value = 0.0000
Reject Null Hypothesis that sample comes from a normal distribution at alpha = 0.05

Sample Distribution Skew = -9.9593  Kurtosis = 41.7558
Gaussian Distribution Skew = 0.1485  Kurtosis = 2.2870
Laplacian Distribution Skew = 2.2486  Kurtosis = 92.6014


    \end{Verbatim}

    \begin{center}
    \adjustimage{max size={0.9\linewidth}{0.9\paperheight}}{output_17_1.png}
    \end{center}
    { \hspace*{\fill} \\}
    
    From the test results, we fail to reject the null hypothesis that the
Bach audio sample is from a normal distribution. Therefore, we can be
confident that this data is not Gaussian. It also appears that the Bach
audio follows a Laplacian distribution due to the high Kurtosis
(\href{https://en.wikipedia.org/wiki/Kurtosis}{this is also known as
super-Gaussianity}). A
\href{https://en.wikipedia.org/wiki/Laplace_distribution}{Laplacian
distribution} has excess positive kurtosis and is called
``leptokurtic''. This distribution has a sharper peak and more weight in
the tails than a standard Gaussian. A Laplacian Distribution is sparse,
meaning that most of the values are concentrated at the mean with a few
significant outliers, lending the distibution wider tails and a sharper
peak than a Gaussian distribution.

    \hypertarget{test-speech-sample}{%
\subsection{Test Speech Sample}\label{test-speech-sample}}

    \begin{Verbatim}[commandchars=\\\{\}]
{\color{incolor}In [{\color{incolor}9}]:} \PY{n}{check\PYZus{}gaussianity}\PY{p}{(}\PY{n}{speech\PYZus{}sample}\PY{p}{)}
\end{Verbatim}


    \begin{Verbatim}[commandchars=\\\{\}]
Normal Test p-value = 0.0000
Reject Null Hypothesis that sample comes from a normal distribution at alpha = 0.05

Sample Distribution Skew = 15.9990  Kurtosis = 84.8716
Gaussian Distribution Skew = -0.2531  Kurtosis = 0.0370
Laplacian Distribution Skew = 1.7515  Kurtosis = 95.4982


    \end{Verbatim}

    \begin{center}
    \adjustimage{max size={0.9\linewidth}{0.9\paperheight}}{output_20_1.png}
    \end{center}
    { \hspace*{\fill} \\}
    
    The speech audio sample also does not follow a normal distribution from
the results of the test. We can therefore move forward with attempting
to separate out the samples from a linear combination of the two
independent, non-Gaussian audio samples.

    \hypertarget{mix-bach-and-speech-samples}{%
\subsection{Mix Bach and Speech
Samples}\label{mix-bach-and-speech-samples}}

The signal we will analyze is a linear combination of the two indepedent
sources created by the equation

\[X = AS\]

X is the signal, A is the mixing matrix, and S is the matrix of sources.
In this case, the sources are mixed together using a random mixing
matrix.

    \begin{Verbatim}[commandchars=\\\{\}]
{\color{incolor}In [{\color{incolor}10}]:} \PY{n}{samples} \PY{o}{=} \PY{n}{audionorm}\PY{p}{(}\PY{p}{[}\PY{n}{bach\PYZus{}sample}\PY{p}{,} \PY{n}{speech\PYZus{}sample}\PY{p}{]}\PY{p}{)}
         
         \PY{c+c1}{\PYZsh{} Standardize samples}
         \PY{n}{samples} \PY{o}{=} \PY{p}{(}\PY{n}{samples}\PY{o}{.}\PY{n}{T} \PY{o}{/} \PY{n}{samples}\PY{o}{.}\PY{n}{std}\PY{p}{(}\PY{n}{axis}\PY{o}{=}\PY{l+m+mi}{1}\PY{p}{)}\PY{p}{)}\PY{o}{.}\PY{n}{T}
         
         \PY{c+c1}{\PYZsh{} Mix samples in linear combination with random mixing matrix}
         \PY{n}{X}\PY{p}{,} \PY{n}{true\PYZus{}A} \PY{o}{=} \PY{n}{simpleMixer}\PY{p}{(}\PY{n}{samples}\PY{p}{)}
         
         \PY{c+c1}{\PYZsh{} Plot the mixed data}
         \PY{n}{drawDataWithMixingMatrix}\PY{p}{(}\PY{n}{X}\PY{p}{,} \PY{n}{true\PYZus{}A}\PY{p}{)}
\end{Verbatim}


    \begin{center}
    \adjustimage{max size={0.9\linewidth}{0.9\paperheight}}{output_23_0.png}
    \end{center}
    { \hspace*{\fill} \\}
    
    \hypertarget{visualize-samples-and-mixed-signal}{%
\subsection{Visualize Samples and Mixed
Signal}\label{visualize-samples-and-mixed-signal}}

To get a sense of what the mixing matrix is doing, we can look at the
individual samples and the combined signal.

    \begin{Verbatim}[commandchars=\\\{\}]
{\color{incolor}In [{\color{incolor}11}]:} \PY{k}{def} \PY{n+nf}{plot\PYZus{}samples\PYZus{}mixed}\PY{p}{(}\PY{n}{samples}\PY{p}{,} \PY{n}{signal}\PY{p}{)}\PY{p}{:}
             \PY{c+c1}{\PYZsh{} Independent Samples}
             \PY{n}{plt}\PY{o}{.}\PY{n}{figure}\PY{p}{(}\PY{n}{figsize}\PY{o}{=}\PY{p}{(}\PY{l+m+mi}{8}\PY{p}{,} \PY{l+m+mi}{6}\PY{p}{)}\PY{p}{)}
             \PY{n}{plt}\PY{o}{.}\PY{n}{plot}\PY{p}{(}\PY{n}{samples}\PY{p}{[}\PY{l+m+mi}{0}\PY{p}{,} \PY{p}{:}\PY{p}{]}\PY{p}{,} \PY{n}{label} \PY{o}{=} \PY{l+s+s1}{\PYZsq{}}\PY{l+s+s1}{Sample 1}\PY{l+s+s1}{\PYZsq{}}\PY{p}{)}
             \PY{n}{plt}\PY{o}{.}\PY{n}{plot}\PY{p}{(}\PY{n}{samples}\PY{p}{[}\PY{l+m+mi}{1}\PY{p}{,} \PY{p}{:}\PY{p}{]}\PY{p}{,} \PY{n}{label} \PY{o}{=} \PY{l+s+s1}{\PYZsq{}}\PY{l+s+s1}{Sample 2}\PY{l+s+s1}{\PYZsq{}}\PY{p}{)}
             \PY{n}{plt}\PY{o}{.}\PY{n}{legend}\PY{p}{(}\PY{p}{)}
             \PY{n}{plt}\PY{o}{.}\PY{n}{title}\PY{p}{(}\PY{l+s+s1}{\PYZsq{}}\PY{l+s+s1}{Samples}\PY{l+s+s1}{\PYZsq{}}\PY{p}{,} \PY{n}{size} \PY{o}{=} \PY{l+m+mi}{18}\PY{p}{)}
             \PY{n}{plt}\PY{o}{.}\PY{n}{show}\PY{p}{(}\PY{p}{)}\PY{p}{;} 
             
             \PY{c+c1}{\PYZsh{} Mixed signal}
             \PY{n}{plt}\PY{o}{.}\PY{n}{figure}\PY{p}{(}\PY{n}{figsize}\PY{o}{=}\PY{p}{(}\PY{l+m+mi}{8}\PY{p}{,} \PY{l+m+mi}{6}\PY{p}{)}\PY{p}{)}
             \PY{n}{plt}\PY{o}{.}\PY{n}{plot}\PY{p}{(}\PY{n}{signal}\PY{p}{[}\PY{l+m+mi}{0}\PY{p}{,} \PY{p}{:}\PY{p}{]}\PY{p}{,} \PY{n}{alpha} \PY{o}{=} \PY{l+m+mf}{1.0}\PY{p}{,} \PY{n}{label} \PY{o}{=} \PY{l+s+s1}{\PYZsq{}}\PY{l+s+s1}{Mixed 1}\PY{l+s+s1}{\PYZsq{}}\PY{p}{)}
             \PY{n}{plt}\PY{o}{.}\PY{n}{plot}\PY{p}{(}\PY{n}{signal}\PY{p}{[}\PY{l+m+mi}{1}\PY{p}{,} \PY{p}{:}\PY{p}{]}\PY{p}{,} \PY{n}{alpha} \PY{o}{=} \PY{l+m+mf}{1.0}\PY{p}{,} \PY{n}{label} \PY{o}{=} \PY{l+s+s1}{\PYZsq{}}\PY{l+s+s1}{Mixed 2}\PY{l+s+s1}{\PYZsq{}}\PY{p}{)}
             \PY{n}{plt}\PY{o}{.}\PY{n}{legend}\PY{p}{(}\PY{p}{)}
             \PY{n}{plt}\PY{o}{.}\PY{n}{title}\PY{p}{(}\PY{l+s+s1}{\PYZsq{}}\PY{l+s+s1}{Mixed Signal}\PY{l+s+s1}{\PYZsq{}}\PY{p}{,} \PY{n}{size} \PY{o}{=} \PY{l+m+mi}{18}\PY{p}{)}
             \PY{n}{plt}\PY{o}{.}\PY{n}{show}\PY{p}{(}\PY{p}{)}\PY{p}{;}
\end{Verbatim}


    \begin{Verbatim}[commandchars=\\\{\}]
{\color{incolor}In [{\color{incolor}12}]:} \PY{n}{plot\PYZus{}samples\PYZus{}mixed}\PY{p}{(}\PY{n}{samples}\PY{p}{,} \PY{n}{X}\PY{p}{)}
\end{Verbatim}


    \begin{center}
    \adjustimage{max size={0.9\linewidth}{0.9\paperheight}}{output_26_0.png}
    \end{center}
    { \hspace*{\fill} \\}
    
    \begin{center}
    \adjustimage{max size={0.9\linewidth}{0.9\paperheight}}{output_26_1.png}
    \end{center}
    { \hspace*{\fill} \\}
    
    \hypertarget{verify-ica-implementation}{%
\section{Verify ICA Implementation}\label{verify-ica-implementation}}

Before separating the audio sources, we will verify the ICA
implementation using synthetic data. The data is generated from a
Laplacian Distribution and combined with a defined mixing matrix so we
know that the overall signal is a linear combination of non-Gaussian
components and the ICA method should function very well on this data.

    \hypertarget{generate-synthetic-laplacian-data}{%
\subsection{Generate Synthetic Laplacian
Data}\label{generate-synthetic-laplacian-data}}

    \begin{Verbatim}[commandchars=\\\{\}]
{\color{incolor}In [{\color{incolor}13}]:} \PY{c+c1}{\PYZsh{} Generate a mixed signal for a number of samples and}
         \PY{c+c1}{\PYZsh{} defined mixing matrix}
         \PY{k}{def} \PY{n+nf}{generate\PYZus{}laplacian\PYZus{}data}\PY{p}{(}\PY{n}{n\PYZus{}samples}\PY{p}{,} \PY{n}{mixing}\PY{p}{)}\PY{p}{:}
             \PY{c+c1}{\PYZsh{} Set seed for reproducible results}
             \PY{n}{np}\PY{o}{.}\PY{n}{random}\PY{o}{.}\PY{n}{seed}\PY{p}{(}\PY{n}{seed} \PY{o}{=} \PY{l+m+mi}{50}\PY{p}{)}
             
             \PY{c+c1}{\PYZsh{} Laplacian distributions}
             \PY{n}{s1} \PY{o}{=} \PY{n}{np}\PY{o}{.}\PY{n}{random}\PY{o}{.}\PY{n}{laplace}\PY{p}{(}\PY{n}{size} \PY{o}{=} \PY{n}{n\PYZus{}samples}\PY{p}{)}
             \PY{n}{s2} \PY{o}{=} \PY{n}{np}\PY{o}{.}\PY{n}{random}\PY{o}{.}\PY{n}{laplace}\PY{p}{(}\PY{n}{size} \PY{o}{=} \PY{n}{n\PYZus{}samples}\PY{p}{)}
             
             \PY{c+c1}{\PYZsh{} Combine into one array}
             \PY{n}{S} \PY{o}{=} \PY{n}{np}\PY{o}{.}\PY{n}{array}\PY{p}{(}\PY{p}{[}\PY{n}{s1}\PY{p}{,} \PY{n}{s2}\PY{p}{]}\PY{p}{)}
             
             \PY{c+c1}{\PYZsh{} Mix samples}
             \PY{n}{generated\PYZus{}data} \PY{o}{=} \PY{n}{mixing}\PY{o}{.}\PY{n}{dot}\PY{p}{(}\PY{n}{S}\PY{p}{)}
             
             \PY{k}{return} \PY{n}{generated\PYZus{}data}\PY{p}{,} \PY{n}{S}
\end{Verbatim}


    \begin{Verbatim}[commandchars=\\\{\}]
{\color{incolor}In [{\color{incolor}14}]:} \PY{c+c1}{\PYZsh{} Defined mixing matrix}
         \PY{n}{verify\PYZus{}mixing} \PY{o}{=} \PY{n}{np}\PY{o}{.}\PY{n}{array}\PY{p}{(}\PY{p}{[}\PY{p}{[}\PY{o}{\PYZhy{}}\PY{l+m+mi}{1}\PY{p}{,} \PY{l+m+mi}{1}\PY{p}{]}\PY{p}{,}
                                   \PY{p}{[}\PY{l+m+mi}{2}\PY{p}{,} \PY{l+m+mi}{2}\PY{p}{]}\PY{p}{]}\PY{p}{)}
         
         \PY{c+c1}{\PYZsh{} Signal for verification}
         \PY{n}{verify\PYZus{}signal}\PY{p}{,} \PY{n}{verify\PYZus{}samples} \PY{o}{=} \PY{n}{generate\PYZus{}laplacian\PYZus{}data}\PY{p}{(}\PY{n}{n\PYZus{}samples} \PY{o}{=} \PY{l+m+mi}{10000}\PY{p}{,} \PY{n}{mixing} \PY{o}{=} \PY{n}{verify\PYZus{}mixing}\PY{p}{)}
         
         \PY{c+c1}{\PYZsh{} Visualize the verification data}
         \PY{n}{drawDataWithMixingMatrix}\PY{p}{(}\PY{n}{verify\PYZus{}signal}\PY{p}{,} \PY{n}{verify\PYZus{}mixing}\PY{p}{)}
\end{Verbatim}


    \begin{center}
    \adjustimage{max size={0.9\linewidth}{0.9\paperheight}}{output_30_0.png}
    \end{center}
    { \hspace*{\fill} \\}
    
    \begin{Verbatim}[commandchars=\\\{\}]
{\color{incolor}In [{\color{incolor}15}]:} \PY{n}{check\PYZus{}gaussianity}\PY{p}{(}\PY{n}{verify\PYZus{}samples}\PY{p}{[}\PY{l+m+mi}{0}\PY{p}{,} \PY{p}{:}\PY{p}{]}\PY{p}{)}
\end{Verbatim}


    \begin{Verbatim}[commandchars=\\\{\}]
Normal Test p-value = 0.0000
Reject Null Hypothesis that sample comes from a normal distribution at alpha = 0.05

Sample Distribution Skew = -2.8428  Kurtosis = 26.3515
Gaussian Distribution Skew = -0.8293  Kurtosis = -0.4725
Laplacian Distribution Skew = 0.5005  Kurtosis = 28.1637


    \end{Verbatim}

    \begin{center}
    \adjustimage{max size={0.9\linewidth}{0.9\paperheight}}{output_31_1.png}
    \end{center}
    { \hspace*{\fill} \\}
    
    \hypertarget{visualize-generated-validation-data}{%
\subsubsection{Visualize Generated Validation
Data}\label{visualize-generated-validation-data}}

    \begin{Verbatim}[commandchars=\\\{\}]
{\color{incolor}In [{\color{incolor}16}]:} \PY{n}{plot\PYZus{}samples\PYZus{}mixed}\PY{p}{(}\PY{n}{verify\PYZus{}samples}\PY{p}{,} \PY{n}{verify\PYZus{}signal}\PY{p}{)}
\end{Verbatim}


    \begin{center}
    \adjustimage{max size={0.9\linewidth}{0.9\paperheight}}{output_33_0.png}
    \end{center}
    { \hspace*{\fill} \\}
    
    \begin{center}
    \adjustimage{max size={0.9\linewidth}{0.9\paperheight}}{output_33_1.png}
    \end{center}
    { \hspace*{\fill} \\}
    
    \hypertarget{implement-ica-on-validation-data}{%
\subsection{Implement ICA on Validation
Data}\label{implement-ica-on-validation-data}}

    \begin{Verbatim}[commandchars=\\\{\}]
{\color{incolor}In [{\color{incolor}17}]:} \PY{c+c1}{\PYZsh{} Use developed implementation of ICA, signal must be transposed}
         \PY{n}{mixing}\PY{p}{,} \PY{n}{sources}\PY{p}{,} \PY{n}{mean} \PY{o}{=} \PY{n}{perform\PYZus{}fastica}\PY{p}{(}\PY{n}{verify\PYZus{}signal}\PY{o}{.}\PY{n}{T}\PY{p}{,} \PY{n}{n\PYZus{}components} \PY{o}{=} \PY{l+m+mi}{2}\PY{p}{,} \PY{n}{print\PYZus{}negentropy}\PY{o}{=}\PY{k+kc}{True}\PY{p}{)}
\end{Verbatim}


    \begin{Verbatim}[commandchars=\\\{\}]
Iteration: 0 Increase in Negentropy: 0.0423.
Iteration: 1 Increase in Negentropy: 0.0085.
Iteration: 2 Increase in Negentropy: 0.0000.

    \end{Verbatim}

    \hypertarget{examine-estimated-mixing-matrix}{%
\subsection{Examine Estimated Mixing
Matrix}\label{examine-estimated-mixing-matrix}}

    \begin{Verbatim}[commandchars=\\\{\}]
{\color{incolor}In [{\color{incolor}18}]:} \PY{n}{compareMixingMatrix}\PY{p}{(}\PY{n}{verify\PYZus{}signal}\PY{p}{,} \PY{n}{verify\PYZus{}mixing}\PY{p}{,} \PY{n}{mixing}\PY{p}{)}
\end{Verbatim}


    \begin{center}
    \adjustimage{max size={0.9\linewidth}{0.9\paperheight}}{output_37_0.png}
    \end{center}
    { \hspace*{\fill} \\}
    
    \begin{center}
    \adjustimage{max size={0.9\linewidth}{0.9\paperheight}}{output_37_1.png}
    \end{center}
    { \hspace*{\fill} \\}
    
    \begin{Verbatim}[commandchars=\\\{\}]
{\color{incolor}In [{\color{incolor}19}]:} \PY{n+nb}{print}\PY{p}{(}\PY{l+s+s1}{\PYZsq{}}\PY{l+s+s1}{True Mixing Matrix}\PY{l+s+se}{\PYZbs{}n}\PY{l+s+s1}{\PYZsq{}}\PY{p}{)}
         \PY{n+nb}{print}\PY{p}{(}\PY{n}{verify\PYZus{}mixing}\PY{p}{)}
         \PY{n+nb}{print}\PY{p}{(}\PY{l+s+s1}{\PYZsq{}}\PY{l+s+se}{\PYZbs{}n}\PY{l+s+s1}{Estimated Mixing Matrix}\PY{l+s+se}{\PYZbs{}n}\PY{l+s+s1}{\PYZsq{}}\PY{p}{)}
         \PY{n+nb}{print}\PY{p}{(}\PY{n}{mixing}\PY{p}{)}
\end{Verbatim}


    \begin{Verbatim}[commandchars=\\\{\}]
True Mixing Matrix

[[-1  1]
 [ 2  2]]

Estimated Mixing Matrix

[[ 138.34822884 -140.37567214]
 [ 289.40506692  269.28916873]]

    \end{Verbatim}

    Although the numbers in the mixing matrix are not the same, the
independent components have been correctly identified. The
\href{https://stats.stackexchange.com/questions/30348/is-it-acceptable-to-reverse-a-sign-of-a-principal-component-score}{signs
on the independent components} may be reversed from the actual values
because the signs are arbitrary. Also, ICA is a non-deterministic
algorithm, so the exact mixing matrix will vary every run. What we can
see from the figure is that the FastICA method has identified the
independent components.

    \hypertarget{examine-estimated-sources}{%
\subsection{Examine Estimated Sources}\label{examine-estimated-sources}}

We can also look at the estimated sources from ICA. These should be the
same as the original within a sign change and a constant scaling factor.

    \begin{Verbatim}[commandchars=\\\{\}]
{\color{incolor}In [{\color{incolor}20}]:} \PY{k}{def} \PY{n+nf}{plot\PYZus{}samples\PYZus{}estimates}\PY{p}{(}\PY{n}{samples}\PY{p}{,} \PY{n}{estimates}\PY{p}{)}\PY{p}{:}
             \PY{c+c1}{\PYZsh{} Independent Samples}
             \PY{n}{plt}\PY{o}{.}\PY{n}{figure}\PY{p}{(}\PY{n}{figsize}\PY{o}{=}\PY{p}{(}\PY{l+m+mi}{8}\PY{p}{,} \PY{l+m+mi}{6}\PY{p}{)}\PY{p}{)}
             \PY{n}{plt}\PY{o}{.}\PY{n}{plot}\PY{p}{(}\PY{n}{samples}\PY{p}{[}\PY{l+m+mi}{0}\PY{p}{,} \PY{p}{:}\PY{p}{]}\PY{p}{,} \PY{n}{label} \PY{o}{=} \PY{l+s+s1}{\PYZsq{}}\PY{l+s+s1}{Sample 1}\PY{l+s+s1}{\PYZsq{}}\PY{p}{)}
             \PY{n}{plt}\PY{o}{.}\PY{n}{plot}\PY{p}{(}\PY{n}{samples}\PY{p}{[}\PY{l+m+mi}{1}\PY{p}{,} \PY{p}{:}\PY{p}{]}\PY{p}{,} \PY{n}{label} \PY{o}{=} \PY{l+s+s1}{\PYZsq{}}\PY{l+s+s1}{Sample 2}\PY{l+s+s1}{\PYZsq{}}\PY{p}{)}
             \PY{n}{plt}\PY{o}{.}\PY{n}{legend}\PY{p}{(}\PY{p}{)}
             \PY{n}{plt}\PY{o}{.}\PY{n}{title}\PY{p}{(}\PY{l+s+s1}{\PYZsq{}}\PY{l+s+s1}{Ground Truth Samples}\PY{l+s+s1}{\PYZsq{}}\PY{p}{,} \PY{n}{size} \PY{o}{=} \PY{l+m+mi}{18}\PY{p}{)}
             \PY{n}{plt}\PY{o}{.}\PY{n}{show}\PY{p}{(}\PY{p}{)}\PY{p}{;} 
             
             \PY{c+c1}{\PYZsh{} Mixed signal}
             \PY{n}{plt}\PY{o}{.}\PY{n}{figure}\PY{p}{(}\PY{n}{figsize}\PY{o}{=}\PY{p}{(}\PY{l+m+mi}{8}\PY{p}{,} \PY{l+m+mi}{6}\PY{p}{)}\PY{p}{)}
             \PY{n}{plt}\PY{o}{.}\PY{n}{plot}\PY{p}{(}\PY{n}{estimates}\PY{p}{[}\PY{l+m+mi}{0}\PY{p}{,} \PY{p}{:}\PY{p}{]}\PY{p}{,} \PY{n}{alpha} \PY{o}{=} \PY{l+m+mf}{1.0}\PY{p}{,} \PY{n}{label} \PY{o}{=} \PY{l+s+s1}{\PYZsq{}}\PY{l+s+s1}{Mixed 1}\PY{l+s+s1}{\PYZsq{}}\PY{p}{)}
             \PY{n}{plt}\PY{o}{.}\PY{n}{plot}\PY{p}{(}\PY{n}{estimates}\PY{p}{[}\PY{l+m+mi}{1}\PY{p}{,} \PY{p}{:}\PY{p}{]}\PY{p}{,} \PY{n}{alpha} \PY{o}{=} \PY{l+m+mf}{1.0}\PY{p}{,} \PY{n}{label} \PY{o}{=} \PY{l+s+s1}{\PYZsq{}}\PY{l+s+s1}{Mixed 2}\PY{l+s+s1}{\PYZsq{}}\PY{p}{)}
             \PY{n}{plt}\PY{o}{.}\PY{n}{legend}\PY{p}{(}\PY{p}{)}
             \PY{n}{plt}\PY{o}{.}\PY{n}{title}\PY{p}{(}\PY{l+s+s1}{\PYZsq{}}\PY{l+s+s1}{Estimated Samples}\PY{l+s+s1}{\PYZsq{}}\PY{p}{,} \PY{n}{size} \PY{o}{=} \PY{l+m+mi}{18}\PY{p}{)}
             \PY{n}{plt}\PY{o}{.}\PY{n}{show}\PY{p}{(}\PY{p}{)}\PY{p}{;}
\end{Verbatim}


    \begin{Verbatim}[commandchars=\\\{\}]
{\color{incolor}In [{\color{incolor}21}]:} \PY{n}{plot\PYZus{}samples\PYZus{}estimates}\PY{p}{(}\PY{n}{verify\PYZus{}samples}\PY{p}{,} \PY{n}{sources}\PY{o}{.}\PY{n}{T}\PY{p}{)}
\end{Verbatim}


    \begin{center}
    \adjustimage{max size={0.9\linewidth}{0.9\paperheight}}{output_42_0.png}
    \end{center}
    { \hspace*{\fill} \\}
    
    \begin{center}
    \adjustimage{max size={0.9\linewidth}{0.9\paperheight}}{output_42_1.png}
    \end{center}
    { \hspace*{\fill} \\}
    
    From the above visuals, we can be confident that our ICA implementation
works on a signal that is a linear combination of non-Gaussian
components. It is able to identify the mixing matrix and the original
sources from which the signal is derived.

    \hypertarget{test-ica-on-audio-data}{%
\section{Test ICA on Audio Data}\label{test-ica-on-audio-data}}

Now it's time to apply the FastICA implementation to the mixed audio
samples. We should be able to find an appropriate mixing matrix and
separate out the sources.

    \begin{Verbatim}[commandchars=\\\{\}]
{\color{incolor}In [{\color{incolor}22}]:} \PY{n}{audio\PYZus{}mixing}\PY{p}{,} \PY{n}{audio\PYZus{}sources}\PY{p}{,} \PY{n}{audio\PYZus{}mean} \PY{o}{=} \PY{n}{perform\PYZus{}fastica}\PY{p}{(}\PY{n}{X}\PY{o}{.}\PY{n}{T}\PY{p}{,} \PY{n}{n\PYZus{}components}\PY{o}{=}\PY{l+m+mi}{2}\PY{p}{,} \PY{n}{print\PYZus{}negentropy}\PY{o}{=}\PY{k+kc}{True}\PY{p}{)}
\end{Verbatim}


    \begin{Verbatim}[commandchars=\\\{\}]
Iteration: 0 Increase in Negentropy: 0.0078.
Iteration: 1 Increase in Negentropy: 0.0000.

    \end{Verbatim}

    \hypertarget{visualize-actual-and-estimated-mixing-matrices}{%
\subsection{Visualize Actual and Estimated Mixing
Matrices}\label{visualize-actual-and-estimated-mixing-matrices}}

    \begin{Verbatim}[commandchars=\\\{\}]
{\color{incolor}In [{\color{incolor}23}]:} \PY{n}{compareMixingMatrix}\PY{p}{(}\PY{n}{X}\PY{p}{,} \PY{n}{true\PYZus{}A}\PY{p}{,} \PY{n}{audio\PYZus{}mixing}\PY{p}{)}
\end{Verbatim}


    \begin{center}
    \adjustimage{max size={0.9\linewidth}{0.9\paperheight}}{output_47_0.png}
    \end{center}
    { \hspace*{\fill} \\}
    
    \begin{center}
    \adjustimage{max size={0.9\linewidth}{0.9\paperheight}}{output_47_1.png}
    \end{center}
    { \hspace*{\fill} \\}
    
    \begin{Verbatim}[commandchars=\\\{\}]
{\color{incolor}In [{\color{incolor}24}]:} \PY{n+nb}{print}\PY{p}{(}\PY{l+s+s1}{\PYZsq{}}\PY{l+s+s1}{True Mixing Matrix}\PY{l+s+se}{\PYZbs{}n}\PY{l+s+s1}{\PYZsq{}}\PY{p}{)}
         \PY{n+nb}{print}\PY{p}{(}\PY{n}{true\PYZus{}A}\PY{p}{)}
         \PY{n+nb}{print}\PY{p}{(}\PY{l+s+s1}{\PYZsq{}}\PY{l+s+se}{\PYZbs{}n}\PY{l+s+s1}{Estimated Mixing Matrix}\PY{l+s+se}{\PYZbs{}n}\PY{l+s+s1}{\PYZsq{}}\PY{p}{)}
         \PY{n+nb}{print}\PY{p}{(}\PY{n}{audio\PYZus{}mixing}\PY{p}{)}
\end{Verbatim}


    \begin{Verbatim}[commandchars=\\\{\}]
True Mixing Matrix

[[0.37454012 0.95071431]
 [0.73199394 0.59865848]]

Estimated Mixing Matrix

[[-345.98250801 -138.58964824]
 [-218.00245656 -267.77563995]]

    \end{Verbatim}

    As mentioned previously, the signs on the Independent Components are
arbitrary. However, we can see that the algorithm was able to identify
the independent components as seen in the mixing matrix.

We also see from the printed information that the Negentropy of the
independent components decreased over the iterations. The FastICA
implementation maximizes the non-Gaussianity of the Independent
Components.

    \hypertarget{visualize-actual-and-estimated-sources}{%
\subsection{Visualize Actual and Estimated
Sources}\label{visualize-actual-and-estimated-sources}}

    \begin{Verbatim}[commandchars=\\\{\}]
{\color{incolor}In [{\color{incolor}25}]:} \PY{n}{plot\PYZus{}samples\PYZus{}estimates}\PY{p}{(}\PY{n}{samples}\PY{p}{,} \PY{n}{audio\PYZus{}sources}\PY{o}{.}\PY{n}{T}\PY{p}{)}
\end{Verbatim}


    \begin{center}
    \adjustimage{max size={0.9\linewidth}{0.9\paperheight}}{output_51_0.png}
    \end{center}
    { \hspace*{\fill} \\}
    
    \begin{center}
    \adjustimage{max size={0.9\linewidth}{0.9\paperheight}}{output_51_1.png}
    \end{center}
    { \hspace*{\fill} \\}
    
    Again, we see that within a sign change and a constant scaling factor,
the ICA method separated out the sources. Given the mixing matrix plot
and the source plot, we can have confidence that our implementation is
working as expected.

    \hypertarget{conclusions}{%
\section{Conclusions}\label{conclusions}}

In this notebook, we tested our developed implementation of Indepedent
Component Analysis both on synthetic data and on audio samples. For both
cases, the ICA method was able to identify the independent components as
observed in the estimated mixing matrix and the sources. We did see that
the mixing matrix does not have the same numbers as the original because
there may be an
\href{http://people.wku.edu/david.neal/307/Unit2/LinearComb.pdf}{infinite
number of solutions} for the mixing matrix because the signal is a
linear combination of the sources
(\href{https://math.stackexchange.com/questions/204768/linear-combinations-and-solutions}{a
linear transformation gives either one, zero, or infinite solutions.})
There will always be
\href{http://cs229.stanford.edu/notes/cs229-notes11.pdf}{ambiguity both
in the mixing matrix and the sources from independent component
analysis} because there is no method to recover the exact scaling of the
independent sources. This answers a question from the previous notebook
where we saw differences in sign changes and in scaling between the
results and the known values. Overall, this assignment has demonstrated
a fast implementation of Independent Component Analysis and verified its
use in finding the sources in signals that are composed of linear
combinations of non-Gaussian indepedent components.


    % Add a bibliography block to the postdoc
    
    
    
    \end{document}
